\documentclass[9pt]{elife} % no lineno or draft
% Nice citations
%\usepackage{natbib}
% Set margins
%\usepackage[margin=2.5cm]{geometry}
% Multilingual support
%\usepackage[english]{babel}
% Nice mathematics and Left-right harpoons for kinetic equations
\usepackage{amsmath,mathtools}
% Control over maketitle
%\usepackage{titling,titlesec}
% Ability to use colour in text
\usepackage{color}
% For the \degree symbol
\usepackage{textcomp,gensymb}
% Allow includegraphics and wrapped figures
\usepackage{graphicx,wrapfig,subfigure}
\usepackage[outercaption]{sidecap}
\usepackage{csquotes}
% For full page figures:
%\usepackage[CaptionAfterwards]{fitpage}

%% Trick to define a language alias and permit language = {en} in the .bib file.
% From: http://tex.stackexchange.com/questions/199254/babel-define-language-synonym
\usepackage{letltxmacro}
\LetLtxMacro{\ORIGselectlanguage}{\selectlanguage}
\makeatletter
\DeclareRobustCommand{\selectlanguage}[1]{%
  \@ifundefined{alias@\string#1}
    {\ORIGselectlanguage{#1}}
    {\begingroup\edef\x{\endgroup
       \noexpand\ORIGselectlanguage{\@nameuse{alias@#1}}}\x}%
}
\newcommand{\definelanguagealias}[2]{%
  \@namedef{alias@#1}{#2}%
}
\makeatother
\definelanguagealias{en}{english}
\definelanguagealias{eng}{english}
%% End language alias trick

%% Any aliases here
\newcommand{\mb}[1]{\mathbf{#1}} % this won't work?
% Emphasis and bold.
\newcommand{\e}{\emph}
\newcommand{\code}[1]{\textsf{#1}}
\newcommand{\dvrg}{\nabla\vcdot\nabla}

\definecolor{colcmnt}        {rgb} {0.7098, 0.0745, 0.0431}
\newcommand{\redcmnt}[1]{\textcolor{colcmnt}{#1}}

%% END aliases

% Custom font defs
% fontsize is \fontsize{fontsize}{linespacesize}
\def\authorListFont{\fontsize{11}{11} }
\def\corrAuthorFont{\fontsize{10}{10} }
\def\affiliationListFont{\fontsize{11}{11}\itshape }
\def\titleFont{\fontsize{14}{11} \bfseries }
\def\textFont{\fontsize{11}{11} }
\def\sectionHdrFont{\fontsize{11}{11}\bfseries}
\def\bibFont{\fontsize{10}{10} }
\def\captionFont{\fontsize{10}{10} }

% Caption font size to be small.
%\usepackage[font=small,labelfont=bf]{caption}

\title {none}

\author[1]{Sebastian~S.~James}
\author[1*]{Stuart~P.~Wilson}
\affil[1]{Department of Psychology, The University of Sheffield, Sheffield, United Kingdom.}
\corr{s.wilson@sheffield.ac.uk}{SW}
%% END OF PREAMBLE

\begin{document}

\textbf{Figure 1.} Schematic diagrams of retinotectal mapping and gene expression. Coloured lines indicate that temporal retinal ganglion cells project to the rostral part of the contra-lateral tectum; nasal cells to caudal tectum; dorsal cells find their way to the lateral tectum and ventral retinal cells project to the medial tectum.
%
\textbf{A} Tectal ligand expression $L_i$. Tectal ligand expression patterns are assumed to increase exponentially in one direction only across the tectal surface. $L_0$ (or ephrin A) increases in a caudal to rostral direction (i.e. with increasing $y$; $x$ and $y$ directions are indicated with arrows). $L_2$ increases in the opposing sense, with $-y$. $L_1$ and $L_3$ form counter-gradients in the $\pm x$ directions.
\emph{L: lateral, M: medial, C: caudal, R: rostral}.
%
\textbf{B} Retinal receptor and ligand expressions. Four receptors, $r_i$, are expressed across the retina. $r_0$ increases as an exponential function of $-x$ ; $r_1$ increases with an exponential function of $-y$. $r_2$ and $r_3$ have the opposite sense and increase with positive $x$ and $y$, respectively. Retinal ligand expression, $l_i$, is expressed in two complimentary gradients \citep{hornberger_modulation_1999}.
 \emph{N: nasal, T: temporal, D: dorsal, V: ventral}.
%

\bibliography{RetinoTectal}

\end{document}
