\documentclass[9pt]{elife} % no lineno or draft
% Nice citations
%\usepackage{natbib}
% Set margins
%\usepackage[margin=2.5cm]{geometry}
% Multilingual support
%\usepackage[english]{babel}
% Nice mathematics and Left-right harpoons for kinetic equations
\usepackage{amsmath,mathtools}
% Control over maketitle
%\usepackage{titling,titlesec}
% Ability to use colour in text
\usepackage{color}
% For the \degree symbol
\usepackage{textcomp,gensymb}
% Allow includegraphics and wrapped figures
\usepackage{graphicx,wrapfig,subfigure}
\usepackage[outercaption]{sidecap}
\usepackage{csquotes}
% For full page figures:
%\usepackage[CaptionAfterwards]{fitpage}

%% Trick to define a language alias and permit language = {en} in the .bib file.
% From: http://tex.stackexchange.com/questions/199254/babel-define-language-synonym
\usepackage{letltxmacro}
\LetLtxMacro{\ORIGselectlanguage}{\selectlanguage}
\makeatletter
\DeclareRobustCommand{\selectlanguage}[1]{%
  \@ifundefined{alias@\string#1}
    {\ORIGselectlanguage{#1}}
    {\begingroup\edef\x{\endgroup
       \noexpand\ORIGselectlanguage{\@nameuse{alias@#1}}}\x}%
}
\newcommand{\definelanguagealias}[2]{%
  \@namedef{alias@#1}{#2}%
}
\makeatother
\definelanguagealias{en}{english}
\definelanguagealias{eng}{english}
%% End language alias trick

%% Any aliases here
\newcommand{\mb}[1]{\mathbf{#1}} % this won't work?
% Emphasis and bold.
\newcommand{\e}{\emph}
\newcommand{\code}[1]{\textsf{#1}}
\newcommand{\dvrg}{\nabla\vcdot\nabla}

\definecolor{colcmnt}        {rgb} {0.7098, 0.0745, 0.0431}
\newcommand{\redcmnt}[1]{\textcolor{colcmnt}{#1}}

%% END aliases

% Custom font defs
% fontsize is \fontsize{fontsize}{linespacesize}
\def\authorListFont{\fontsize{11}{11} }
\def\corrAuthorFont{\fontsize{10}{10} }
\def\affiliationListFont{\fontsize{11}{11}\itshape }
\def\titleFont{\fontsize{14}{11} \bfseries }
\def\textFont{\fontsize{11}{11} }
\def\sectionHdrFont{\fontsize{11}{11}\bfseries}
\def\bibFont{\fontsize{10}{10} }
\def\captionFont{\fontsize{10}{10} }

% Caption font size to be small.
%\usepackage[font=small,labelfont=bf]{caption}

\title {none}

\author[1]{Sebastian~S.~James}
\author[1*]{Stuart~P.~Wilson}
\affil[1]{Department of Psychology, The University of Sheffield, Sheffield, United Kingdom.}
\corr{s.wilson@sheffield.ac.uk}{SW}
%% END OF PREAMBLE

\begin{document}

\section{Introduction}
The axons of retinal ganglion cells form an ordered map of the retina in the
optic tectum. Surgical and genetic experiments have revealed several key
mechanisms that contribute to retinotectal map development, and computational
modelling studies have demonstrated how maps can be shaped by their
interactions. Models derived from Sperry's classic chemoaffinity theory can
reproduce retinotopy, and Gierer later showed how an extension to include
competition between growing axons could also account for a subset of the
disordered maps that have resulted from surgical manipulation. Further
extensions have each accounted for additional datapoints, but none has been
able to explain the full range of maps observed experimentally. Here we
present a unified model of retinotectal development, assuming only local
interactions between growing axons, which can account for the full range of
maps resulting from surgical manipulation. This model extends the `potential
functions' of Gierer's original to describe self-organisation across a 2D
cortical sheet, and it represents a coordination of axonal growth by
orthogonal pairs of opposing ligand gradients in the tectum and corresponding
receptor gradients in the retina. In addition, we present progress towards
accounting for the full range of maps that have resulted from genetic
manipulation, by incorporating a mechanistic description of receptor ligand
interactions.

One important class of theories assumes that map formation is driven by
chemotaxis, influenced by pairs of anti-parallel expression gradients in the
tectum, with the balance of influences in each axon determined by a receptor
density specific to its origin in the retina
(\citealp{Gierer1983,Gierer1987,Simpson2011}; see also
\citealp{Karbowski2004,James2020}). A second class of theories assumes that a
single gradient is primarily responsible for chemotaxis and that maps reflect
additional processes such as competition \citep{Triplett2011}, Hebbian
activity \citep{Tsigankov2006,Tsigankov2010}, or marker induction
\citep{Prestige1975,Willshaw2006}.

A key test of the validity of recent models is their ability to account for
data from a set of genetic manipulation studies by
\cite{brown_topographic_2000} and \cite{reber_relative_2004}, in which axons
from a subset of retinal ganglia cells treated with a Eph-A3 knock-in form a
separate, rostrally shifted topographic map, which becomes indistinct from
that formed by untreated cells, at a `collapse point' on the rostral
side. \cite{Sterratt2013} found that a mixture of counter-gradients and an
additional competition term provided the best account of these data. A
follow-up study in which a range of different computational models were
systematically compared \citep{hjorth_quantitative_2015} found that one
focused on competition by Koulakov \citep{Triplett2011} could best account for
several such experimental findings, and that the other types of model could
not explain the collapse point.

The explanatory power of Koulakov's `switching' models stems from their
simplicity and degree of abstraction, revealing the map pattern that optimises
the constraints represented by a given energy function. However, while
switching models are evaluated through a numerical procedure (i.e., iterative
switching), that procedure should not necessarily be interpreted as an
explicit representation of the process of development
\citep{Wilson2015}.

We turned back to mechanistic models, taking as our starting point the model of
\cite{Simpson2011}, which incorporates chemotaxis, competitive interactions
and receptor-ligand interactions. This model used a global supervisor to
provide the chemotactic component of axon movement. We set about determining
whether we could re-cast the model into one in which simulated axon growth is
guided by only local information. If such a model could account for the full
range of experimental results, it would represent the most complete account of
retinotectal map self-organization to date.

\bibliography{RetinoTectal}

\end{document}
