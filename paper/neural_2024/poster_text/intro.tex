\documentclass[9pt]{elife} % no lineno or draft
% Nice citations
%\usepackage{natbib}
% Set margins
%\usepackage[margin=2.5cm]{geometry}
% Multilingual support
%\usepackage[english]{babel}
% Nice mathematics and Left-right harpoons for kinetic equations
\usepackage{amsmath,mathtools}
% Control over maketitle
%\usepackage{titling,titlesec}
% Ability to use colour in text
\usepackage{color}
% For the \degree symbol
\usepackage{textcomp,gensymb}
% Allow includegraphics and wrapped figures
\usepackage{graphicx,wrapfig,subfigure}
\usepackage[outercaption]{sidecap}
\usepackage{csquotes}
% For full page figures:
%\usepackage[CaptionAfterwards]{fitpage}

%% Trick to define a language alias and permit language = {en} in the .bib file.
% From: http://tex.stackexchange.com/questions/199254/babel-define-language-synonym
\usepackage{letltxmacro}
\LetLtxMacro{\ORIGselectlanguage}{\selectlanguage}
\makeatletter
\DeclareRobustCommand{\selectlanguage}[1]{%
  \@ifundefined{alias@\string#1}
    {\ORIGselectlanguage{#1}}
    {\begingroup\edef\x{\endgroup
       \noexpand\ORIGselectlanguage{\@nameuse{alias@#1}}}\x}%
}
\newcommand{\definelanguagealias}[2]{%
  \@namedef{alias@#1}{#2}%
}
\makeatother
\definelanguagealias{en}{english}
\definelanguagealias{eng}{english}
%% End language alias trick

%% Any aliases here
\newcommand{\mb}[1]{\mathbf{#1}} % this won't work?
% Emphasis and bold.
\newcommand{\e}{\emph}
\newcommand{\code}[1]{\textsf{#1}}
\newcommand{\dvrg}{\nabla\vcdot\nabla}

\definecolor{colcmnt}        {rgb} {0.7098, 0.0745, 0.0431}
\newcommand{\redcmnt}[1]{\textcolor{colcmnt}{#1}}

%% END aliases

% Custom font defs
% fontsize is \fontsize{fontsize}{linespacesize}
\def\authorListFont{\fontsize{11}{11} }
\def\corrAuthorFont{\fontsize{10}{10} }
\def\affiliationListFont{\fontsize{11}{11}\itshape }
\def\titleFont{\fontsize{14}{11} \bfseries }
\def\textFont{\fontsize{11}{11} }
\def\sectionHdrFont{\fontsize{11}{11}\bfseries}
\def\bibFont{\fontsize{10}{10} }
\def\captionFont{\fontsize{10}{10} }

% Caption font size to be small.
%\usepackage[font=small,labelfont=bf]{caption}

\title {none}

\author[1]{Sebastian~S.~James}
\author[1*]{Stuart~P.~Wilson}
\affil[1]{Department of Psychology, The University of Sheffield, Sheffield, United Kingdom.}
\corr{s.wilson@sheffield.ac.uk}{SW}
%% END OF PREAMBLE

\begin{document}

%\section{Introduction}

\subsection{A summary of this work}

\begin{itemize}
\item Retinal cells grow axons that form an ordered map in the optic tectum
\item Sperry, Gierer, Prestige, Willshaw and many others have all sought to model this
  developmental process to better understand how brains wire themselves up
\item Surgical and genetic manipulation experiments have provided a wide range
  of developmental responses that these models should be able to describe
\item We want to make a model based on local interactions that can account for
  all of the experimental results
\end{itemize}

\subsection{A brief history of retinotectal studies}

(For reviews see \citealp{Prestige1975,Goodhill2005})

\begin{itemize}
\item 1940s: Sperry's surgical experimental and theoretical work
  \citep{sperry_reestablishment_1942,sperry_visuomotor_1943,sperry_chemoaffinity_1963}
\item 1970s: Marker-induction models \citep{Prestige1975,Willshaw2006}
\item 1980s: Chemotaxis models \citep{Gierer1983,Gierer1987}, branch arrow
  models \citep{overton_extended_1982}
\item 1990s: Discovery of the ephrin signalling molecules \citep{nakamoto_topographically_1996}
\item 2000s: Important genetic manipulation studies
  \citep{brown_topographic_2000,reber_relative_2004}; Hebbian activity models \citep{Tsigankov2006,Tsigankov2010}
\item 2010s: Competition models \citep{Triplett2011,Simpson2011}; comparative
  studies \citep{hjorth_quantitative_2015}
\item Recently: Re-evaluation of chemoaffinity \citep{Sterratt2013,naoki_revisiting_2017}
\end{itemize}

One of the more abstract models \citep{Triplett2011} was shown by
\cite{hjorth_quantitative_2015} to be the only one able to explain the `EphA3
collapse point'. However, although the model is evaluated numerically (by
iterative switching), this procedure is not necessarily an explicit
representation of the process of development \citep{Wilson2015}. We wanted to
find a mechanistic model which can be directly related to the behaviour of
developing axon processes. Starting with a modified version of \cite{James2020},
and influenced by the \cite{Simpson2011}
we built a Gierer-like gradient climbing model in which we attempted to
express all the interactions of the axon branches in terms of receptor-ligand
signalling.


\bibliography{RetinoTectal}

\end{document}
