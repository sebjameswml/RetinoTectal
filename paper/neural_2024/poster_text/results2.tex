\documentclass[9pt]{elife} % no lineno or draft
% Nice citations
%\usepackage{natbib}
% Set margins
%\usepackage[margin=2.5cm]{geometry}
% Multilingual support
%\usepackage[english]{babel}
% Nice mathematics and Left-right harpoons for kinetic equations
\usepackage{amsmath,mathtools}
% Control over maketitle
%\usepackage{titling,titlesec}
% Ability to use colour in text
\usepackage{color}
% For the \degree symbol
\usepackage{textcomp,gensymb}
% Allow includegraphics and wrapped figures
\usepackage{graphicx,wrapfig,subfigure}
\usepackage[outercaption]{sidecap}
\usepackage{csquotes}
% For full page figures:
%\usepackage[CaptionAfterwards]{fitpage}

%% Trick to define a language alias and permit language = {en} in the .bib file.
% From: http://tex.stackexchange.com/questions/199254/babel-define-language-synonym
\usepackage{letltxmacro}
\LetLtxMacro{\ORIGselectlanguage}{\selectlanguage}
\makeatletter
\DeclareRobustCommand{\selectlanguage}[1]{%
  \@ifundefined{alias@\string#1}
    {\ORIGselectlanguage{#1}}
    {\begingroup\edef\x{\endgroup
       \noexpand\ORIGselectlanguage{\@nameuse{alias@#1}}}\x}%
}
\newcommand{\definelanguagealias}[2]{%
  \@namedef{alias@#1}{#2}%
}
\makeatother
\definelanguagealias{en}{english}
\definelanguagealias{eng}{english}
%% End language alias trick

%% Any aliases here
\newcommand{\mb}[1]{\mathbf{#1}} % this won't work?
% Emphasis and bold.
\newcommand{\e}{\emph}
\newcommand{\code}[1]{\textsf{#1}}
\newcommand{\dvrg}{\nabla\vcdot\nabla}

\definecolor{colcmnt}        {rgb} {0.7098, 0.0745, 0.0431}
\newcommand{\redcmnt}[1]{\textcolor{colcmnt}{#1}}

%% END aliases

% Custom font defs
% fontsize is \fontsize{fontsize}{linespacesize}
\def\authorListFont{\fontsize{11}{11} }
\def\corrAuthorFont{\fontsize{10}{10} }
\def\affiliationListFont{\fontsize{11}{11}\itshape }
\def\titleFont{\fontsize{14}{11} \bfseries }
\def\textFont{\fontsize{11}{11} }
\def\sectionHdrFont{\fontsize{11}{11}\bfseries}
\def\bibFont{\fontsize{10}{10} }
\def\captionFont{\fontsize{10}{10} }

% Caption font size to be small.
%\usepackage[font=small,labelfont=bf]{caption}

\title {none}

\author[1]{Sebastian~S.~James}
\author[1*]{Stuart~P.~Wilson}
\affil[1]{Department of Psychology, The University of Sheffield, Sheffield, United Kingdom.}
\corr{s.wilson@sheffield.ac.uk}{SW}
%% END OF PREAMBLE

\begin{document}

\subsection*{A gradient based model of chemotaxis}

\citet{simpson_simple_2011} chose a placeholder mechanism for the chemotaxis effect, using a global supervisor to set $\mathbf{G}$ to a vector which would always point to the predetermined target location for each branch.
In order to make this a local model we replaced the global supervisor with an interaction between RGC receptors and ligands expressed on the tectum. We used forms for the ligand expression, $L(\mathbf{x})$, which matched those used for the RGC receptor expression.

We assumed a purely linear receptor binding model, and set the chemotactic movement vector of the branch $b$ at location $\mathbf{x}$ on the tectum to be

\begin{equation}\label{e:G}
\mathbf{G} = m_{\!_G}\,\sum_i^N F_i\,r_{i,b} \nabla L_i(\mathbf{x})
\end{equation}
%
where $r_{i,b}$ is the receptor expression on branch $b$ for ligand-receptor pair $i$, $L_i$ is the expression of ligand $i$ on the tectum and $F_i$ denotes the direction of the interaction induced when a molecule of ligand $i$ binds to a receptor $i$ molecule.
$F_i$ takes the value $-1$ for a repulsive interaction or $1$ for an attractive interaction.
%
We assumed that all receptor-ligand signalled interactions are repulsive ($F_i=-1, \forall i$), as for EphA-ephrinA coupling \citep{drescher_vitro_1995,nakamoto_topographically_1996}.
%
$m_{\!_G}$ is a scalar parameter which controls how much movement is generated for a given level of receptor-ligand gradient signalling.

Each of the $M$ RGC growth cones carry a set of $N$ receptors, $r_i$, indexed by $i$ and expressed at levels determined by the cell soma's location on the retinal grid.
%
We defined 4 receptor expression gradients arranged in orthogonal pairs with the gradient of $r_0$ being orthogonal to that of $r_1$. $r_2$ was orthogonal to $r_3$ (Fig.\,1B). Thus the model is of \emph{differential chemotaxis}.

%Cells on the tectum express ligands, $L_i$, for the retinal receptors, also in
%orthogonal pairs of gradients.  Several studies model tectal ligand expression
%with exponential functions \citep{koulakov_stochastic_2004}. Although the
%experimental evidence to support this is lean, we also set $L_i$ to the same
%exponential used for retinal expression.

\bibliography{RetinoTectal}

\end{document}
