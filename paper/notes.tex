% This is an example of using latex for a paper/report of specified
% size/layout. It's useful if you want to provide a PDF that looks
% like it was made in a normal word processor.

% While writing, don't stop for errors
\nonstopmode

% Use the article doc class, with an 11 pt basic font size
\documentclass[11pt, a4paper]{article}

% Makes the main font Nimbus Roman, a Times New Roman lookalike:
%\usepackage{mathptmx}% http://ctan.org/pkg/mathptmx
% OR use this for proper Times New Roman (from msttcorefonts package
% on Ubuntu). Use xelatex instead of pdflatex to compile:
\usepackage{fontspec}
\usepackage{xltxtra}
\usepackage{xunicode}
\defaultfontfeatures{Scale=MatchLowercase,Mapping=tex-text}
\setmainfont{Times New Roman}

% Set margins
\usepackage[margin=2.5cm]{geometry}

% Multilingual support
\usepackage[english]{babel}

% Nice mathematics
\usepackage{amsmath}

% Left right harpoons for kinetic equations
\usepackage{mathtools}

% Control over maketitle
\usepackage{titling}

% Section styling
\usepackage{titlesec}

% Ability to use colour in text
\usepackage[usenames]{color}

% For the \degree symbol
\usepackage{gensymb}

% Allow includegraphics and nice wrapped figures
\usepackage{graphicx}
\usepackage{wrapfig}
\usepackage[outercaption]{sidecap}

% Set formats using titlesec
\titleformat*{\section}{\bfseries\rmfamily}
\titleformat*{\subsection}{\bfseries\itshape\rmfamily}

% thetitle is the number of the section. This sets the distance from
% the number to the section text.
\titlelabel{\thetitle.\hskip0.3em\relax}

% Set title spacing with titlesec, too.  The first {1.0ex plus .2ex
% minus .7ex} sets the spacing above the section title. The second
% {-1.0ex plus 0.2ex} sets the spacing the section title to the
% paragraph.
\titlespacing{\section}{0pc}{1.0ex plus .2ex minus .7ex}{-1.1ex plus 0.2ex}

%% Trick to define a language alias and permit language = {en} in the .bib file.
% From: http://tex.stackexchange.com/questions/199254/babel-define-language-synonym
\usepackage{letltxmacro}
\LetLtxMacro{\ORIGselectlanguage}{\selectlanguage}
\makeatletter
\DeclareRobustCommand{\selectlanguage}[1]{%
  \@ifundefined{alias@\string#1}
    {\ORIGselectlanguage{#1}}
    {\begingroup\edef\x{\endgroup
       \noexpand\ORIGselectlanguage{\@nameuse{alias@#1}}}\x}%
}
\newcommand{\definelanguagealias}[2]{%
  \@namedef{alias@#1}{#2}%
}
\makeatother
\definelanguagealias{en}{english}
\definelanguagealias{eng}{english}
%% End language alias trick

%% Any aliases here
\newcommand{\mb}[1]{\mathbf{#1}} % this won't work?
% Emphasis and bold.
\newcommand{\e}{\emph}
\newcommand{\mycite}[1]{\cite{#1}}
\newcommand{\code}[1]{\textsf{#1}}
\newcommand{\dvrg}{\nabla\vcdot\nabla}
%% END aliases

% Custom font defs
% fontsize is \fontsize{fontsize}{linespacesize}
\def\authorListFont{\fontsize{11}{11} }
\def\corrAuthorFont{\fontsize{10}{10} }
\def\affiliationListFont{\fontsize{11}{11}\itshape }
\def\titleFont{\fontsize{14}{11} \bfseries }
\def\textFont{\fontsize{11}{11} }
\def\sectionHdrFont{\fontsize{11}{11}\bfseries}
\def\bibFont{\fontsize{10}{10} }
\def\captionFont{\fontsize{10}{10} }

% Caption font size to be small.
\usepackage[font=small,labelfont=bf]{caption}

% Make a dot for the dot product, call it vcdot for 'vector calculus
% dot'. Bigger than \cdot, smaller than \bullet.
\makeatletter
\newcommand*\vcdot{\mathpalette\vcdot@{.35}}
\newcommand*\vcdot@[2]{\mathbin{\vcenter{\hbox{\scalebox{#2}{$\m@th#1\bullet$}}}}}
\makeatother

\def\firstAuthorLast{James}

% Affiliations
\def\Address{\\
\affiliationListFont Adaptive Behaviour Research Group, Department of Psychology,
  The University of Sheffield, Sheffield, UK \\
}

% The Corresponding Author should be marked with an asterisk. Provide
% the exact contact address (this time including street name and city
% zip code) and email of the corresponding author
\def\corrAuthor{Seb James}
\def\corrAddress{Department of Psychology, The University of Sheffield,
  Western Bank, Sheffield, S10 2TP, UK}
\def\corrEmail{seb.james@sheffield.ac.uk}

% Figure out the font for the author list..
\def\Authors{\authorListFont Sebastian James\\[1 ex]  \Address \\
  \corrAuthorFont $^{*}$ Correspondence: \corrEmail}

% No page numbering please
\pagenumbering{gobble}

% A trick to get the bibliography to show up with 1. 2. etc in place
% of [1], [2] etc.:
\makeatletter
\renewcommand\@biblabel[1]{#1.}
\makeatother

% reduce separation between bibliography items if not using natbib:
\let\OLDthebibliography\thebibliography
\renewcommand\thebibliography[1]{
  \OLDthebibliography{#1}
  \setlength{\parskip}{0pt}
  \setlength{\itemsep}{0pt plus 0.3ex}
}

% Set correct font for bibliography (doesn't work yet)
%\renewcommand*{\bibfont}{\bibFont}

% No paragraph indenting to match the VPH format
\setlength{\parindent}{0pt}

% Skip a line after paragraphs
\setlength{\parskip}{0.5\baselineskip}
\onecolumn

% titling definitions
\pretitle{\begin{center}\titleFont}
\posttitle{\par\end{center}\vskip 0em}
\preauthor{ % Fonts are set within \Authors
        \vspace{-1.1cm} % Bring authors up towards title
        \begin{center}
        \begin{tabular}[t]{c}
}
\postauthor{\end{tabular}\par\end{center}}

% Define title, empty date and authors
\title {
  Modelling the Retinotectal projection - can competition provide a stopping mechanism?
}
\date{} % No date please
\author{\Authors}

%% END OF PREAMBLE

\begin{document}

\setlength{\droptitle}{-1.8cm} % move the title up a suitable amount
\maketitle

\vspace{-1.8cm} % HACK bring the introduction up towards the title. It
                % would be better to do this with titling in \maketitle

%%%%%%%%%%%%%%%%%%%%%%%%%%%%%%%%%%%%%%%%%%%%%%%%%%%%%%%%%%%%%%%%%%%%%%%%%%%%%%%
\section{Introduction}

I want to see if the competition mechanism which we demonstrated can form the
murine barrel pattern~\cite{james_modelling_2020} can also help explain the
arrangement of the Retinotectal projection.

The Retinotectal (RT) projection is thought to use pairs of orthogonal
gradients in the retina to specify the eventual location of synapses made on
the surface of the tectum. Similar orthogonal gradients in the tectum provide
a coordinate system which allows the axons to match their prespecified
destination with an actual location on the tectum. Although the Ephrins have
been shown to provide the guidance which sends axons \e{towards} their
destination~\cite{feldheim_genetic_2000}, no mechanism has yet been discovered
which actually \e{halts} the axonal growth cones in the correct location.

The main difference which I envisage between the RT and whisker barrel model
systems is in the way competition is applied to axons growing towards the
destination tissue. In the barrel field, I (notionally) assigned to a `group'
of axons the same interaction parameter, if they originated from the same
thalamic barreloid. The Retiontectal system is, in contrast, continuous;
although I am hypothesizing that the RT axons obtain an interaction parameter
from their location in the retina, just as I did for axons originating in
different barreloids, I will allow each axon to have a unique interaction
parameter, and expect that this leads to the continuous arrangement of axons
expected in the tectum.

To test whether the competetive model (assuming it does reproduce the
retinotopic neural arrangement) is useful, I can consider the following
manipulations:

I'd like to make a comparison with a system in which the stopping criterion is
an absolute measurement of the expression level of the Ephrins in the
tectum. I plan to introduce noise to the measurement that the growth cone
makes of the Ephrin expression level. This will introduce stochasticity in the
movement of the growth cones. The competition mechanism will be supported if
it is more robust to noise than an absolute stopping mechanism.

I will also then need to examine whether the competition mechanism is robust
to the experimental manipulations of the sort which Sperry carried
out~\cite{sperry_chemoaffinity_1963,goodhill_retinotectal_1999,goodhill_development_2005}.

\section{Model}

The initial model is very similar to that in the Barrels paper. However,
instead of providing interaction parameters ($\gamma_{i,j}$) in a configuration
file, I allow the config to specify $N$ neurons originating in the retina; I
automatically arrange these in a radially symmetric pattern, and then
automatically obtain the $gamma$ parameters from the neuron locations within
the pattern.

\subsection{Configurations}

In the config directory, there are numerous basic configs, such as N12.json,
N24.json and so on. I find I have to modify some of the parameters as N
increases. Larger N requires smaller hexes to allow a suitable pattern to
develop. Larger N also generally requires a smaller diffusion constant
$D$. I'm experimenting with $\epsilon$, the competition parameter, which, if
too high causes the retinotopy to degrade during the simulation.


%
% BIBLIOGRAPHY
%
\selectlanguage{English}
\bibliographystyle{abbrvnotitle}
\bibliography{RetinoTectal}

\end{document}
