% This is an example of using latex for a paper/report of specified
% size/layout. It's useful if you want to provide a PDF that looks
% like it was made in a normal word processor.

% While writing, don't stop for errors
\nonstopmode

% Use the article doc class, with an 11 pt basic font size
\documentclass[11pt, a4paper]{article}

% Makes the main font Nimbus Roman, a Times New Roman lookalike:
%\usepackage{mathptmx}% http://ctan.org/pkg/mathptmx
% OR use this for proper Times New Roman (from msttcorefonts package
% on Ubuntu). Use xelatex instead of pdflatex to compile:
\usepackage{fontspec}
\usepackage{xltxtra}
\usepackage{xunicode}
\defaultfontfeatures{Scale=MatchLowercase,Mapping=tex-text}
\setmainfont{Times New Roman}

% Set margins
\usepackage[margin=2.5cm]{geometry}

% Multilingual support
\usepackage[english]{babel}

% Nice mathematics
\usepackage{amsmath}

% Left right harpoons for kinetic equations
\usepackage{mathtools}

% Control over maketitle
\usepackage{titling}

% Section styling
\usepackage{titlesec}

% Ability to use colour in text
\usepackage[usenames]{color}

% For the \degree symbol
\usepackage{gensymb}

% Allow includegraphics and nice wrapped figures
\usepackage{graphicx}
\usepackage{wrapfig}
\usepackage[outercaption]{sidecap}

% Nice quotes
\usepackage{csquotes}

% Set formats using titlesec
\titleformat*{\section}{\bfseries\rmfamily}
\titleformat*{\subsection}{\bfseries\itshape\rmfamily}

% thetitle is the number of the section. This sets the distance from
% the number to the section text.
\titlelabel{\thetitle.\hskip0.3em\relax}

% Set title spacing with titlesec, too.  The first {1.0ex plus .2ex
% minus .7ex} sets the spacing above the section title. The second
% {-1.0ex plus 0.2ex} sets the spacing the section title to the
% paragraph.
\titlespacing{\section}{0pc}{1.0ex plus .2ex minus .7ex}{-1.1ex plus 0.2ex}

%% Trick to define a language alias and permit language = {en} in the .bib file.
% From: http://tex.stackexchange.com/questions/199254/babel-define-language-synonym
\usepackage{letltxmacro}
\LetLtxMacro{\ORIGselectlanguage}{\selectlanguage}
\makeatletter
\DeclareRobustCommand{\selectlanguage}[1]{%
  \@ifundefined{alias@\string#1}
    {\ORIGselectlanguage{#1}}
    {\begingroup\edef\x{\endgroup
       \noexpand\ORIGselectlanguage{\@nameuse{alias@#1}}}\x}%
}
\newcommand{\definelanguagealias}[2]{%
  \@namedef{alias@#1}{#2}%
}
\makeatother
\definelanguagealias{en}{english}
\definelanguagealias{eng}{english}
%% End language alias trick

%% Any aliases here
\newcommand{\mb}[1]{\mathbf{#1}} % this won't work?
% Emphasis and bold.
\newcommand{\e}{\emph}
\newcommand{\mycite}[1]{\cite{#1}}
\newcommand{\code}[1]{\textsf{#1}}
\newcommand{\dvrg}{\nabla\vcdot\nabla}
%% END aliases

% Custom font defs
% fontsize is \fontsize{fontsize}{linespacesize}
\def\authorListFont{\fontsize{11}{11} }
\def\corrAuthorFont{\fontsize{10}{10} }
\def\affiliationListFont{\fontsize{11}{11}\itshape }
\def\titleFont{\fontsize{14}{11} \bfseries }
\def\textFont{\fontsize{11}{11} }
\def\sectionHdrFont{\fontsize{11}{11}\bfseries}
\def\bibFont{\fontsize{10}{10} }
\def\captionFont{\fontsize{10}{10} }

% Caption font size to be small.
\usepackage[font=small,labelfont=bf]{caption}

% Make a dot for the dot product, call it vcdot for 'vector calculus
% dot'. Bigger than \cdot, smaller than \bullet.
\makeatletter
\newcommand*\vcdot{\mathpalette\vcdot@{.35}}
\newcommand*\vcdot@[2]{\mathbin{\vcenter{\hbox{\scalebox{#2}{$\m@th#1\bullet$}}}}}
\makeatother

\def\firstAuthorLast{James}

% Affiliations
\def\Address{\\
\affiliationListFont Adaptive Behaviour Research Group, Department of Psychology,
  The University of Sheffield, Sheffield, UK \\
}

% The Corresponding Author should be marked with an asterisk. Provide
% the exact contact address (this time including street name and city
% zip code) and email of the corresponding author
\def\corrAuthor{Seb James}
\def\corrAddress{Department of Psychology, The University of Sheffield,
  Western Bank, Sheffield, S10 2TP, UK}
\def\corrEmail{seb.james@sheffield.ac.uk}

% Figure out the font for the author list..
\def\Authors{\authorListFont Sebastian James\\[1 ex]  \Address \\
  \corrAuthorFont $^{*}$ Correspondence: \corrEmail}

% No page numbering please
\pagenumbering{gobble}

% A trick to get the bibliography to show up with 1. 2. etc in place
% of [1], [2] etc.:
\makeatletter
\renewcommand\@biblabel[1]{#1.}
\makeatother

% reduce separation between bibliography items if not using natbib:
\let\OLDthebibliography\thebibliography
\renewcommand\thebibliography[1]{
  \OLDthebibliography{#1}
  \setlength{\parskip}{0pt}
  \setlength{\itemsep}{0pt plus 0.3ex}
}

% Set correct font for bibliography (doesn't work yet)
%\renewcommand*{\bibfont}{\bibFont}

% No paragraph indenting to match the VPH format
\setlength{\parindent}{0pt}

% Skip a line after paragraphs
\setlength{\parskip}{0.5\baselineskip}
\onecolumn

% titling definitions
\pretitle{\begin{center}\titleFont}
\posttitle{\par\end{center}\vskip 0em}
\preauthor{ % Fonts are set within \Authors
        \vspace{-1.1cm} % Bring authors up towards title
        \begin{center}
        \begin{tabular}[t]{c}
}
\postauthor{\end{tabular}\par\end{center}}

% Define title, empty date and authors
\title {
  Modelling the Retinotectal projection - can competition provide a stopping mechanism?
}
\date{} % No date please
\author{\Authors}

%% END OF PREAMBLE

\begin{document}

\setlength{\droptitle}{-1.8cm} % move the title up a suitable amount
\maketitle

\vspace{-1.8cm} % HACK bring the introduction up towards the title. It
                % would be better to do this with titling in \maketitle

%%%%%%%%%%%%%%%%%%%%%%%%%%%%%%%%%%%%%%%%%%%%%%%%%%%%%%%%%%%%%%%%%%%%%%%%%%%%%%%
\section{Introduction}

I want to see if the competition mechanism which we demonstrated can form the
murine barrel pattern~\cite{james_modelling_2020} can also help explain the
arrangement of the Retinotectal projection.

The Retinotectal (RT) projection is thought to use pairs of orthogonal
gradients in the retina to specify the eventual location of synapses made on
the surface of the tectum. Similar orthogonal gradients in the tectum provide
a coordinate system which allows the axons to match their prespecified
destination with an actual location on the tectum. Although the Ephrins have
been shown to provide the guidance which sends axons \e{towards} their
destination~\cite{feldheim_genetic_2000}, no mechanism has yet been discovered
which actually \e{halts} the axonal growth cones in the correct location.

The main difference which I envisage between the RT and whisker barrel model
systems is in the way competition is applied to axons growing towards the
destination tissue. In the barrel field, I (notionally) assigned to a `group'
of axons the same interaction parameter, if they originated from the same
thalamic barreloid. The Retiontectal system is, in contrast, continuous;
although I am hypothesizing that the RT axons obtain an interaction parameter
from their location in the retina, just as I did for axons originating in
different barreloids, I will allow each axon to have a unique interaction
parameter, and expect that this leads to the continuous arrangement of axons
expected in the tectum.

To test whether the competetive model (assuming it does reproduce the
retinotopic neural arrangement) is useful, I can consider the following
manipulations:

I'd like to make a comparison with a system in which the stopping criterion is
an absolute measurement of the expression level of the Ephrins in the
tectum. I plan to introduce noise to the measurement that the growth cone
makes of the Ephrin expression level. This will introduce stochasticity in the
movement of the growth cones. The competition mechanism will be supported if
it is more robust to noise than an absolute stopping mechanism.

I will also then need to examine whether the competition mechanism is robust
to the experimental manipulations of the sort which Sperry carried
out~\cite{sperry_chemoaffinity_1963,goodhill_retinotectal_1999,goodhill_development_2005}.

\section{Model}

The initial model is derived from the one in the Barrels
paper~\cite{james_modelling_2020} which is found
in \code{sim/rd\_barrel.h}. The specialization of \code{RD\_Barrel} (and a
sister class, \code{RetArrange}) can be found in \code{sim/rd\_rettec.h}, with
main code file \code{rettec.cpp}. Here, instead of providing interaction
parameters ($\gamma_{i,j}$) in a configuration file, I use only the parameter
$N$, which in this context is the number of neurons originating in the
retina. I automatically arrange these $N$ retinal neurons in a radially
symmetric pattern (See \code{RetArrange}). I then obtain the $\gamma$
parameters from the neuron locations within the pattern.

\subsection{Configurations}

In the config directory, there are numerous basic configs, such as N12.json,
N24.json and so on. I find I have to modify some of the parameters as N
increases. Larger N requires smaller hexes to allow a suitable pattern to
develop. Larger N also generally requires a smaller diffusion constant
$D$. I'm experimenting with $\epsilon$, the competition parameter, which, if
too high causes the retinotopy to degrade during the simulation.

\subsection{Noise}

I wish to examine the effect of noise on the arrangement of axons on the
tectum that results from axon guidance. I assume that there is no noise in the
locations of the retinal neurons which serve as the origin of the interaction
parameters; this remains a symmetric, ordered pattern.

The hypothesis which the model is based on is that the interaction parameters
are derived from the positions of the neurons on the source tissue (the
retina) and that there are signalling molecule gradients which provide the
required coordinate axes. Thus, the first source of noise would be in the
sampling of these gradients and hence on the interaction parameters which are
carried with the efferent axons to the tectum. This noise would be expected to
be fixed after the axons had grown away from the retina.

On arriving at the tectum, axon growth cones must determine their direction of
movement. The model encapsulates the quantitative hypotheses that growth cone
movement is a combination of diffusion and guidance. The guidance is subject
to sampling error and thus a second source of noise can be introduced to the
model. This noise varies with time during the simulation and can be considered
to have two contributions; one from variations in the true expression level of
the signalling gradient and one from sampling error measuring it. We will
combine these into a single error. \textbf{Or maybe not:} Because the
no-competition model (see Section \ref{sec:nocomp}) has a function
$f(\gamma, \rho)$, which represents the axons' ability to sample the absolute
value of $\rho$, it is important to represent the variability of the true
value of $\rho(\mb{x})$.

The system equations are modified as follows. Starting from the branching equation

\begin{equation} \label{eq:da}
\frac{\partial a_i}{\partial t} = \nabla\vcdot\left(D \nabla a_i-a_i\sum_{j=1}^{M} \gamma_{i,j}\nabla \rho_j(\mb{x})\right) - \frac{\partial c_i}{\partial t} + \chi_i,
\end{equation}

the term for the divergence of $\mb{J}$ can be written:
%
\begin{equation}
  \label{eq:divJ}
  \nabla\vcdot\mb{J}_i(\mb{x},t) = \nabla\vcdot\left(D \nabla a_i-a_i\sum_{j=1}^{M} \gamma_{i,j}\nabla \rho_j(\mb{x})\right).
\end{equation}
%
In the barrels model, each $\rho_j(\mb{x})$ was assumed static, allowing the
definition of a time-independent $\mb{g}_i(\mb{x})$:
%
\begin{equation}
\mb{g}_i(\mb{x}) \equiv \sum_{j=1}^{M} \gamma_{i,j} \nabla\rho_j(\mb{x}).
\end{equation}
%
Here, we represent the time-dependence of the signalling molecule expression
by noting that $\rho$ is now a function of time.  The error which the growth
cones make in sampling $\rho$ and estimating $\nabla\rho_j(\mb{x})$ is
$\xi_{\rho'}(t)$:
%
\begin{equation}
\mb{g}_i(\mb{x},t) \equiv \sum_{j=1}^{M} \gamma_{i,j} \big( \nabla\rho_j(\mb{x},t)
+ \xi_{\rho'}(t) \big).
\end{equation}
%
One problem with this representation is that the equations represent a
population model. Within one spatial element of the discretized solution of
the differential equations, the error, $\xi_{\rho'}(t)$ can have only one value at
time $t$; this value is applied to all the individual axon growth cones that
are represented by that element. Nonetheless, it's a useful, and fairly simple
way to represent the guidance error in the system. In practice, we ignore the
variation in $\rho_j$ and write
%
\begin{equation}
\mb{g}_i(\mb{x},t) \equiv \sum_{j=1}^{M} \gamma_{i,j} \big( \nabla\rho_j(\mb{x})
+ \xi_{\rho'}(t) \big).
\end{equation}

$\xi_{\rho'}(t)$ will be included in the model as a function returning numbers sampled
from a normal distribution with mean 0 and standard deviation $\sigma_{\rho'}$.

The systematic error in $\gamma_{i,j}$ will be applied at setup time. The
symmetric gamma values will first be derived from the symmetric layout of the
retinal neurons, then a normally distributed perturbation will be applied to
each $\gamma_{i,j}$.

For retinal neurons with two-dimensional coordinates $\mb{r}_i$, $\gamma$ is
given (in the absence of noise) by
$\gamma_{i,0} = G\,|\mb{r}_i| \mb{r}_i \cdot \hat{\mb{x}}$ and
$\gamma_{i,1} = G\,|\mb{r}_i| \mb{r}_i \cdot \hat{\mb{y}}$, where
$\hat{\mb{x}}$ and $\hat{\mb{y}}$ are unit vectors, and $G$ is simply a gain
on $\gamma$. With noise, these become
$\gamma_{i,0} = G\,\big(\xi_\gamma |\mb{r}_i| \mb{r}_i \cdot \hat{\mb{x}} \big)$ and
$\gamma_{i,1} = G\,\big(\xi_\gamma |\mb{r}_i| \mb{r}_i \cdot \hat{\mb{y}} \big)$

$\xi_\gamma$ is drawn from a normal distribution with mean 0 and standard
deviation $\sigma_{\gamma}$.

\subsection{Non-competition based stopping mechanism}
\label{sec:nocomp}

Here, we seek to model the `pure' chemospecificity theory in which axons are
guided by orthogonal morphogen gradients until they reach their destination,
which they must be able to sense. The axons, guided by the \emph{gradient} of
the morphogen concentration are assumed also to be able to sample
the \emph{absolute} concentration of the signalling molecules.  A reduction in
branching, and possibly an enhancement of synaptogenesis are triggered when
the concentrations of the two orthogonal morphogens are simultaneously
measured to be within the axon's `stopping' ranges.

To find a way to implement this in a population model similar to the
competitive model, we start from the branching equation (dropping the
competition term, $\chi_i$):

\begin{equation} \label{eq:da_again}
\frac{\partial a_i}{\partial t} = \nabla\vcdot\left(D \nabla a_i-a_i\sum_{j=1}^{M} \gamma_{i,j}\nabla \rho_j(\mb{x})\right) - \frac{\partial c_i}{\partial t},
\end{equation}

Spatial transfer of information in the system can be halted if the flux of
axonal branching (the term in brackets) tends to 0.  This can be achieved with
a function of the gammas and rhos:

\begin{equation} \label{eq:da_stopping}
\frac{\partial a_i}{\partial t} = \nabla\vcdot\left(D \nabla a_i-a_i\sum_{j=1}^{M=2} \gamma_{i,j}\nabla \rho_j(\mb{x})\right)\;f_i\big(\gamma_{i,1},\gamma_{i,2},\tilde{\rho}_{1}(\mb{x}),\tilde{\rho}_{2}(\mb{x})\big) - \frac{\partial c_i}{\partial t},
\end{equation}

Where $f_i$ is a function which permits branching far away from a given axon's
target region and tends to zero inside the target region. It represents an
axon's internal `stopping process'. Note carefully that this internal axonal
function does \emph{not} depend directly on space; it is a function of the
interaction parameters $\gamma$ and of the signalling molecule density,
$\rho$. Note that $\tilde{\rho}_j(\mb{x})$ is the axon's \emph{measurement} of
$\rho_j(\mb{x})$ and may include measurement noise, $\xi_\rho$:
$\tilde{\rho}_j = \rho_j + \xi_\rho$. Given that the morphogen gradients are
orthogonal, $(\tilde{\rho}_1,\tilde{\rho}_2)$ defines a vector field
$\vec{\rho}\,(\mb{x})$. Similarly, we can combine $(\gamma_{i,1},\gamma_{i,2})$
into the vector $\vec{\gamma}$.

This makes it easy to start to build $f_i$ with a two dimensional Gaussian
function:

\begin{equation}\label{eq:g}
g_i\big(\vec{\rho}(\mb{x})\big) = \mathrm{e}^{-\big(\frac{\vec{\rho}(\mb{x})-\vec{\gamma}}{2 w}\big)^2 }
\end{equation}

where $\vec{\gamma}$ gives the location of the peak and the parameter $w$ is
the width of the peak. This is squashed through a sigmoid, with
sharpness parameter $s$ and inverted:

\begin{equation}\label{eq:f}
f_i\big(\vec{\rho}(\mb{x})\big) = 1 - \bigg( \frac{2}{1 + \mathrm{e}^{-s
g_i(\vec{\rho}\,)}} - 1 \bigg)
\end{equation}

In one equation:

\begin{equation}\label{eq:gf}
f_i\big(\vec{\rho}(\mb{x})\big) = 2 - \frac{2}{1
+ \mathrm{e}^{-s \mathrm{e}^{-\big(\frac{\vec{\rho}(\mb{x})-\vec{\gamma}}{2 w}\big)^2 } }}
\end{equation}

\subsection{Allowing the stopping mechanism to enhance connection making}

As well as affecting branching, there is no reason why I couldn't have the
axonal function also affect connection making. Consider the equation governing
the rate of change of connections.

\begin{equation} \label{eq:dc}
\frac{\partial c_i}{\partial t} =-\alpha c_i +\beta  \left(1 - \sum_{j=1}^{N} c_{j}\right)[a_i]^k.
\end{equation}

If $\beta$ were to be modulated by $1 - f_i$, then connection making would
occur much more strongly in the correct location. Would it be too good I
wonder? Answer: Yes, it's highly effective. To reproduce, set fModulatesBeta
to true, prop\_modulated to 1 and prop\_unmodulated to 0.

\subsection{Allowing the stopping mechanism to reduce connection decay}

An alternative to the above, is to slow down connection decay within the
stopping zone.

\section{Results}

\subsection{Wild type}

The operation of the model without modification and with zero or low noise.

\subsection{Manipulations}

Reference \cite{goodhill_development_2005} (from which the headings and quotes
below are taken) gives a list of features which a model of retinotectal
development or regeneration should address.

\subsubsection{Ectopic targeting}

\begin{displayquote}
\emph{Retinal axons entering the tectum via abnormal trajectories can still find
their appropriate termination sites (Finlay et al. 1979b; Harris 1982,
1984). This suggests the presence of vector signals throughout the tectum
pointing axons towards their correct termination sites.}
\end{displayquote}

The orthogonal gradients in the current model which signal growth direction of
axonal growth cones naturally provide the vector signals which ensure that
regardless of entry point, growth cones tend towards their correct
location. Can prove with different distributions of initial branching density
($a(\mb{x},t=0)$).

\subsubsection{Shifting connections}

\begin{displayquote}
\emph{In some fish and amphibians, the retina grows by addition of new neurons
around the ciliary margin, while the tectum grows by addition of cells to its
caudomedial edge (Gaze et al. 1974; Fraser 1983).The retinotectal map remains
ordered throughout this time, indicating that the retinotectal projection is
continually shifting caudally. This suggests that, at least in these species,
either chemoaffinity gradients do not irreversibly specify the map or the
gradients must evolve with time.}
\end{displayquote}

The current model should work under the condition that the underlying domain
is increased in size, with the competition acting to ensure that the entire
domain is filled and the ordering maintained.

To do: implement a HexGrid whose boundary increases with time. That will mean
some tampering with \code{morph::HexGrid}.

\subsubsection{Rotation experiments)}

\begin{displayquote}
\emph{If a Xenopus presumptive tectum is rotated early enough during development,
a map is formed that is normal relative to the whole animal, whereas later
rotations lead to a rotated map (Chung \& Cooke 1978).}
\end{displayquote}

These correspond to \cite{goodhill_retinotectal_1999}, Figs.~2B--D and can
include retinal rotation as well as early/late tectal rotations.

The model as it stands contains an assumption about the orientation of the
guidance gradients. So rotating the tectum early enough means making no change
(given that I have no model for \emph{creating} the guidance gradients).

\subsubsection{Map compression and expansion}

\begin{displayquote}
\emph{The map formed after removal of half the retina initially covers half the
tectum (Attardi \& Sperry 1963), but then gradually expands to fill the whole
tectum (Schmidt et al. 1978). The axon terminal density remains the same
(Schmidt et al.  1978). If the optic nerve is then made to regenerate again,
an expanded map is immediately formed (Schmidt 1978). If half the tectum is
ablated, the regenerated map is compressed into the remaining tectal space
(Yoon 1971; Sharma 1972; Cook 1979; Finlay et al. 1979a).}
\end{displayquote}

I need to examine more carefully the timeline of these experiments.

These correspond to \cite{goodhill_retinotectal_1999}, Figs.~2E and 2F.

\subsubsection{Translocation}

\begin{displayquote}
\emph{A}
\end{displayquote}

These correspond to \cite{goodhill_retinotectal_1999}, Figs.~2I and 2J.

\subsubsection{Compound-eye experiments}

\begin{displayquote}
\emph{B}
\end{displayquote}

This corresponds to \cite{goodhill_retinotectal_1999}, Fig 2H.

%
% BIBLIOGRAPHY
%
\selectlanguage{English}
\bibliographystyle{abbrvnotitle}
\bibliography{RetinoTectal}

\end{document}
