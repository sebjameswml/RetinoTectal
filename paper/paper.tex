% This is an example of using latex for a paper/report of specified
% size/layout. It's useful if you want to provide a PDF that looks
% like it was made in a normal word processor.

% While writing, don't stop for errors
\nonstopmode

% Use the article doc class, with an 11 pt basic font size
\documentclass[11pt, a4paper]{article}

% Makes the main font Nimbus Roman, a Times New Roman lookalike:
%\usepackage{mathptmx}% http://ctan.org/pkg/mathptmx
% OR use this for proper Times New Roman (from msttcorefonts package
% on Ubuntu). Use xelatex instead of pdflatex to compile:
\usepackage{fontspec}
\usepackage{xltxtra}
\usepackage{xunicode}
\defaultfontfeatures{Scale=MatchLowercase,Mapping=tex-text}
\setmainfont{Times New Roman}

% Set margins
\usepackage[margin=2.5cm]{geometry}

% Multilingual support
\usepackage[english]{babel}

% Nice mathematics
\usepackage{amsmath}

% Left right harpoons for kinetic equations
\usepackage{mathtools}

% Control over maketitle
\usepackage{titling}

% Section styling
\usepackage{titlesec}

% Ability to use colour in text
\usepackage[usenames]{color}

% For the \degree symbol
\usepackage{gensymb}

% Allow includegraphics and nice wrapped figures
\usepackage{graphicx}
\usepackage{wrapfig}
\usepackage[outercaption]{sidecap}

% Nice quotes
\usepackage{csquotes}

% Set formats using titlesec
\titleformat*{\section}{\bfseries\rmfamily}
\titleformat*{\subsection}{\bfseries\itshape\rmfamily}

% thetitle is the number of the section. This sets the distance from
% the number to the section text.
\titlelabel{\thetitle.\hskip0.3em\relax}

% Set title spacing with titlesec, too.  The first {1.0ex plus .2ex
% minus .7ex} sets the spacing above the section title. The second
% {-1.0ex plus 0.2ex} sets the spacing the section title to the
% paragraph.
\titlespacing{\section}{0pc}{1.0ex plus .2ex minus .7ex}{-1.1ex plus 0.2ex}

%% Trick to define a language alias and permit language = {en} in the .bib file.
% From: http://tex.stackexchange.com/questions/199254/babel-define-language-synonym
\usepackage{letltxmacro}
\LetLtxMacro{\ORIGselectlanguage}{\selectlanguage}
\makeatletter
\DeclareRobustCommand{\selectlanguage}[1]{%
  \@ifundefined{alias@\string#1}
    {\ORIGselectlanguage{#1}}
    {\begingroup\edef\x{\endgroup
       \noexpand\ORIGselectlanguage{\@nameuse{alias@#1}}}\x}%
}
\newcommand{\definelanguagealias}[2]{%
  \@namedef{alias@#1}{#2}%
}
\makeatother
\definelanguagealias{en}{english}
\definelanguagealias{eng}{english}
%% End language alias trick

%% Any aliases here
\newcommand{\mb}[1]{\mathbf{#1}} % this won't work?
% Emphasis and bold.
\newcommand{\e}{\emph}
\newcommand{\mycite}[1]{\cite{#1}}
\newcommand{\code}[1]{\textsf{#1}}
\newcommand{\dvrg}{\nabla\vcdot\nabla}
%% END aliases

% Custom font defs
% fontsize is \fontsize{fontsize}{linespacesize}
\def\authorListFont{\fontsize{11}{11} }
\def\corrAuthorFont{\fontsize{10}{10} }
\def\affiliationListFont{\fontsize{11}{11}\itshape }
\def\titleFont{\fontsize{14}{11} \bfseries }
\def\textFont{\fontsize{11}{11} }
\def\sectionHdrFont{\fontsize{11}{11}\bfseries}
\def\bibFont{\fontsize{10}{10} }
\def\captionFont{\fontsize{10}{10} }

% Caption font size to be small.
\usepackage[font=small,labelfont=bf]{caption}

% Make a dot for the dot product, call it vcdot for 'vector calculus
% dot'. Bigger than \cdot, smaller than \bullet.
\makeatletter
\newcommand*\vcdot{\mathpalette\vcdot@{.35}}
\newcommand*\vcdot@[2]{\mathbin{\vcenter{\hbox{\scalebox{#2}{$\m@th#1\bullet$}}}}}
\makeatother

\def\firstAuthorLast{James}

% Affiliations
\def\Address{\\
\affiliationListFont Adaptive Behaviour Research Group, Department of Psychology,
  The University of Sheffield, Sheffield, UK \\
}

% The Corresponding Author should be marked with an asterisk. Provide
% the exact contact address (this time including street name and city
% zip code) and email of the corresponding author
\def\corrAuthor{Seb James}
\def\corrAddress{Department of Psychology, The University of Sheffield,
  Western Bank, Sheffield, S10 2TP, UK}
\def\corrEmail{seb.james@sheffield.ac.uk}

% Figure out the font for the author list..
\def\Authors{\authorListFont Sebastian~S.~James, Stuart~P.~Wilson  \Address \\
  \corrAuthorFont $^{*}$ Correspondence: \corrEmail}

% No page numbering please
\pagenumbering{gobble}

% A trick to get the bibliography to show up with 1. 2. etc in place
% of [1], [2] etc.:
\makeatletter
\renewcommand\@biblabel[1]{#1.}
\makeatother

% reduce separation between bibliography items if not using natbib:
\let\OLDthebibliography\thebibliography
\renewcommand\thebibliography[1]{
  \OLDthebibliography{#1}
  \setlength{\parskip}{0pt}
  \setlength{\itemsep}{0pt plus 0.3ex}
}

% Set correct font for bibliography (doesn't work yet)
%\renewcommand*{\bibfont}{\bibFont}

% No paragraph indenting to match the VPH format
\setlength{\parindent}{0pt}

% Skip a line after paragraphs
\setlength{\parskip}{0.5\baselineskip}
\onecolumn

% titling definitions
\pretitle{\begin{center}\titleFont}
\posttitle{\par\end{center}\vskip 0em}
\preauthor{ % Fonts are set within \Authors
        \vspace{-1.1cm} % Bring authors up towards title
        \begin{center}
        \begin{tabular}[t]{c}
}
\postauthor{\end{tabular}\par\end{center}}

% Define title, empty date and authors
\title {
  Competition can provide a stopping mechanism for the retinotectal
  projection \\
  or \\
  Competition provides a stopping mechanism for the chemoaffinity
  theory which is robust to noise \\
  or \\
  Self-organisation of topographically ordered axon connections can take place
  in a noisy environment if axons compete
}
\date{} % No date please
\author{\Authors}

%% END OF PREAMBLE

\begin{document}

\setlength{\droptitle}{-1.8cm} % move the title up a suitable amount
\maketitle

\vspace{-1.8cm} % HACK bring the introduction up towards the title. It
                % would be better to do this with titling in \maketitle

\emph{Abstract here.}

%%%%%%%%%%%%%%%%%%%%%%%%%%%%%%%%%%%%%%%%%%%%%%%%%%%%%%%%%%%%%%%%%%%%%%%%%%%%%%%
\section{Introduction}

The retinotectal projection has proved to be a deep mine of information for
the study of how the cells of the central nervous system are accurately
connected together into functional networks. This projection connects the
light-gathering cells in the retina to movement related cells in the optic
tectum (known as the superior colliculus in mammals). Light sources
originating close to each other in the environment tend to activate retinal
cells situated close together in the eye so that an image of the environment
is formed on the retinal surface. It has been discovered that the topography
of the retina is preserved within brain regions that process this information
such that cells which are adjacent within the retina primarily excite cells
adjacent in the tectum. This indicates that during development there must
exist a mechanism which ensures the correct wiring of the axons which leave
the retina and connect to cells in the tectum.

One reason for the success of the study of the retinotectal projection is its
capacity to be experimentally manipulated. In some non-mammalian species, both
the retina and the tectum can be partially ablated, or even physically
reorganised in-vivo, after which axons regrow to restore the order and
function of the system for the individual animal. This manipulability was
exploited in influential work by R. W. Sperry and co-workers during the
mid-twentieth century, leading to Sperry's 1963 summary of
the \emph{chemoaffinity theory}~\cite{sperry_chemoaffinity_1963} which
proposes the existence of morphogenetic gradients that guide axons to their
destination. The chemoaffinity theory was given robust support by the
discovery of the ephrin ligands and their receptors~\cite{first_ephrin_paper}
which have been shown to form into graded expression fields in the
retina~\cite{ephrin_retina}, tectum~\cite{ephrin_tectum} as well as in other
sensory systems, such as the somatosensory
system~\cite{vanderhaegen_as_in_elife_paper}. The Ephrin ligands have a clear
effect on axonal outgrowth, as shown in in-vitro~\cite{one} and
in-vivo~\cite{another} studies.

It is tempting to consider the retinotectal projection well understood. With a
comprehensive theory supported by a biochemical mechanism, is there anything
left to understand? That question can be answered by reviewing retinotectal
modelling papers.

Mini-review here, which contrasts some of the modelling papers. The upshot is
that a central problem with models is not how the axons know how to get closer
to their destination, but how they know they have \emph{arrived} at their true
destination.

In a recent paper, we proposed a self-organising mechanism, based on
morphogenetic signalling gradients, which can arrange
axons growing from the thalamus into the somatosensory cortex into the well
known murine barrel cortex pattern~\cite{james_modelling_2020}. In
characterising that system, we explored the effect of various types of
noise. We found that the mechanism was robust to noise in the expression of
the signalling molecules over a wide range of amplitudes and length-scales,
and that noise in the interaction parameters, which are obtained by sampling
the signalling molecules in the source tissue (the thalamic barreloid field)
could cause topological defects. The question of noise in axon guidance has
been explored by Goodhill~\cite{goodhill_can_2016}.

\section{Model}

\section{Results}

\section{Discussion}

%
% BIBLIOGRAPHY
%
\selectlanguage{English}
\bibliographystyle{abbrvnotitle}
\bibliography{RetinoTectal}

\end{document}
